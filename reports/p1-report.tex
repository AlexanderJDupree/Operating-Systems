\documentclass[11pt,letterpaper]{report}
\usepackage[latin1]{inputenc}
\usepackage{amsmath}
\usepackage{amsfonts}
\usepackage{amssymb}
\usepackage{graphicx}
\usepackage{color}
\graphicspath{{./images/}{IR}}
%\newcommand{\LF}{}  % turn on to display large format
\ifdefined \LF
\usepackage[left=2.0cm, top=2.0cm, landscape]{geometry}  % for large format landscape
\else
\usepackage[left=2.0cm, top=2.0cm]{geometry}
\fi
\usepackage{fancyhdr}
\pagestyle{fancy}
\fancyhead{}
\lhead{CS333}
\chead{Project 1 Test Report}
\rhead{Alexander DuPree}
\begin{document}
	
\ifdefined \LF
{\Large     % large print start
\fi


  \section*{Testing}
  \noindent
  The following test report documents the tests performed for project one. The test cases and strategies closely follow the project one rubric. 

  \subsection*{Compilation Test}
  This section will show that my project one code compiles with the project flag set to ``1'' and also set to ``0''. 
  
  {\bf Subtest 1}: Compile with  CS333\_PROJECT set to 0. Since the listing is so long, this will require two screen shots.
  
  In the first screen shot, the current date and time is displayed as well as the value for CS333\_PROJECT, verifying that it is set to 0. In the second screen show, the same information is displayed. This is used to show that the two screen shots are from the same compilation sequence.
  
  The expected outcome is that the compilation step will correctly compile with the project flag set to ``0''.
  
  \begin{figure}[h!]
	\centering
	\includegraphics[width=0.8\linewidth]{P1compilePas0-1.jpg}
	\caption[Part 1: Project flag set to 0]{Compilation with CS333\_PROJECT set to 0.}
	\label{fig:P1compileP0-1}
  \end{figure}

  \begin{figure}[h!]
	\centering
	\includegraphics[width=0.8\linewidth]{P1compilePas0-2.jpg}
	\caption[Part 2: Project flag set to 0]{Compilation with CS333\_PROJECT set to 0.}
	\label{fig:P1compileP0-2}
  \end{figure}

  The date in the first and second figures show about 12 seconds of elapsed time. This shows that the two date commands occurred close in time. The grep commands before and after the compilation show that the project flag in the Makefile is set to ``0''. The date commands are executed close in time, the project flag shows the same value before and after the compilation, and the compilation shows no errors. This leads to the conclusion that the project code correctly compiles with the project flag turned off.
  
  This subtest {\bf PASSES}.
  
  {\bf Subtest 2}: Boot compile with  CS333\_PROJECT set to 1. Since the listing is so long, this will require two screen shots.
  
  In the first screen shot, the current date and time is displayed as well as the value for CS333\_PROJECT, verifying that it is set to 1. In the second screen show, the same information is displayed. This is used to show that the two screen shots are from the same compilation sequence.
  
  The expected outcome is that the compilation step will correctly compile with the project flag set to ``1''.

  
\begin{figure}[h!]
	\centering
	\includegraphics[width=0.8\linewidth]{P1compilePas1-1.jpg}
	\caption[Part 1: Project flag set to 0]{Compilation with CS333\_PROJECT set to 0.}
	\label{fig:P1compileP1-1}
\end{figure}

\begin{figure}[h!]
	\centering
	\includegraphics[width=0.8\linewidth]{P1compilePas1-2.jpg}
	\caption[Part 2: Project flag set to 0]{Compilation with CS333\_PROJECT set to 0.}
	\label{fig:P1compileP1-2}
\end{figure}

The date in the first and second figures show about 10 seconds of elapsed time. This shows that the two date commands occurred close in time. The grep commands before and after the compilation show that the project flag in the Makefile is set to ``1''. The date commands are executed close in time, the project flag shows the same value before and after the compilation, and the compilation shows no errors. This leads to the conclusion that the project code correctly compiles with the project flag turned off.

This subtest {\bf PASSES}.

Since each subtest passes and the subtests fully test the objectives, this test {\bf PASSES}.

  \subsection*{PRINT\_SYSCALLS Test}
  This test verifies that my kernel correctly compiles with the flag {\tt PRINT\_SYSCALLS} turned off, set to 0 in the Makefile.
  This test has two subtests: 1) the kernel correctly compiles and boots with the {\tt CS333\_P1}| flag turned off; and
  2) the kernel correctly compiles and boots with the {\tt CS333\_P1}| flag turned on. Since the {\tt PRINT\_SYSCALLS} 
  causes all system calls to be printed along with their return codes, booting to the shell is sufficient as that
  process causes many system calls to be invoked.
  It is expected that no system call information will be printed to the console.
  
  \textbf{Subtest 1}: PRINT\_SYSCALLS  and CS333\_PROJECT set to 0.
  
  \begin{figure}[h!]
    \centering
    \includegraphics[width=0.8\linewidth]{P1ReportPSyscallOff-1.jpg}
    \caption[Syscall Trace Disabled-1]{Makefile with PRINT\_SYSCALLS  and CS333\_PROJECT set to 0.}
    \label{fig:syscalltracedisabled-1}
  \end{figure}

  \begin{figure}[h!]
  	\centering
  	\includegraphics[width=0.8\linewidth]{P1ReportPSyscallOff-2.jpg}
  	\caption[Syscall Trace Disabled-2]{Boot with PRINT\_SYSCALLS  and CS333\_PROJECT set to 0.}
  	\label{fig:syscalltracedisabled-2}
  \end{figure}

  No system call information is displayed on boot with PRINT\_SYSCALLS  and CS333\_PROJECT set to 0, as expected. This subtest {\bf PASSES}.
  
  \pagebreak
  \textbf{Subtest 2}: PRINT\_SYSCALLS set to 0 and CS333\_PROJECT set to 1.
  
  \begin{figure}[h!]
  	\centering
  	\includegraphics[width=0.8\linewidth]{P1ReportPSyscallOff-3.jpg}
  	\caption[Syscall Trace Disabled-3]{Makefile with PRINT\_SYSCALLS set to 0  and CS333\_PROJECT set to 1.}
  	\label{fig:syscalltracedisabled-3}
  \end{figure}
  
  \begin{figure}[h!]
  	\centering
  	\includegraphics[width=0.8\linewidth]{P1ReportPSyscallOff-4.jpg}
  	\caption[Syscall Trace Disabled-4]{Boot with PRINT\_SYSCALLS set to 0 and CS333\_PROJECT set to 1.}
  	\label{fig:syscalltracedisabled-4}
  \end{figure}
  
  No system call information is displayed on boot with PRINT\_SYSCALLS set to 0  and CS333\_PROJECT set to 1, as expected. This subtest {\bf PASSES}.
  
  Both subtests pass. This test therefore \textbf{PASSES}.



  \clearpage
  \subsection*{System Call Tracing Facility}

  This test verifies that my kernel correctly compiles with the flag {\tt PRINT\_SYSCALLS} turned on, set to 1 in the Makefile.
  This test boots the kernel to the shell prompt. The output should contain additional information from the \texttt{PRINT\_SYSCALLS turned off} test; 
  specifically a list of system calls and their return codes should be displayed. This list should closely match the output shown in the
  project description.
  
  \begin{figure}[h!]
    \centering
    \includegraphics[width=0.8\linewidth]{P1ReportPSycallOn-1.jpg}
    \caption[Syscall Trace]{Boot with PRINT\_SYSCALLS set to 1 and CS333\_PROJECT set to 1.}
    \label{fig:syscalltrace-on}
  \end{figure}

  The system call trace correctly displays the invoked system calls and matches the reference output from the project description. Standard
  output is interleaved with the trace output, as expected.

  This test \textbf{PASSES}.

  \clearpage
  \subsection*{Usertests and forktest run with CS333\_P1 flag turned off}

  I tested that xv6 correctly compiles and runs with the {\tt CS333\_P1} flag
  disabled. I set the {\tt CS333\_PROJECT} value in the {\tt Makefile} to 0. I
  then verified this setting and compiled xv6.

  It is expected that xv6 will boot normally and both {\tt usertests} and {\tt
    forktest} will successfully execute. Since {\tt forktest} is executed as
  part of {\tt usertests}, only {\tt usertests} will be executed.

  \begin{figure}[h!]
    \centering
    \includegraphics[width=0.8\linewidth]{P1ReportUsertestsP1Off-1.jpg}
    \caption[CS333\_P1 disabled 1]{CS333\_P1 disabled 1}
    \label{fig:p1off1}
  \end{figure}

  Some output has been elided.

  \begin{figure}[h!]
    \centering
    \includegraphics[width=0.8\linewidth]{P1ReportUsertestsP1Off-2.jpg}
    \caption[CS333\_P1 disabled 2]{CS333\_P1 disabled 2}
    \label{fig:p1off2}
  \end{figure}


  From both figures, we can see that all usertests have passed. Further,
  since {\tt forktest} is run as a part of {\tt usertests}, we know that {\tt
    forktest} has passed.

  This test \textbf{PASSES}.

  \clearpage
  \subsection*{Usertests and forktest run with CS333\_P1 flag turned on}
  I tested that xv6 correctly compiles and runs with the {\tt CS333\_P1} flag
  enabled. I set the {\tt CS333\_PROJECT} value in the {\tt Makefile} to 1,
  compiled and booted xv6 using {\tt make qemu-nox}, and then ran {\tt usertests}. As mentioned above, this is an acceptable test for both 
  {\tt usertests} and {\tt forktest} since {\tt forktest} is run as part of {\tt usertests}.

  It is expected that all tests from {\tt usertests} will pass.

  \begin{figure}[h!]
    \centering
    \includegraphics[width=0.8\linewidth]{P1ReportUsertestsP1On-1.jpg}
    \caption[Usertests with CS333\_P1 enabled]{Usertests with CS333\_P1 enabled}
    \label{fig:usertestsp1on}
  \end{figure}

  From the above figure, we can see that all usertest tests pass. This test \textbf{PASSES}.


  \clearpage
  \subsection*{Date Command}

  This test verifies that my date command works correctly. As implementing timezones was not a requirement and the system clock tracks to
  universal coordinate time (UTC), the expected output would be a close match to running the Linux date command \texttt{date -u}. The test will require 
  1) boot xv6; 2) run the date command under xv6; 3) exit xv6 (I will use the control sequence); and 4) run ``date -u'' at the Linux prompt. Note that the Linux output is expected to display a few seconds later than the xv6 date command as it takes non-zero time to perform the xv6 shutdown and Linux command invocation.

  \begin{figure}[h!]
    \centering
    \includegraphics[width=0.8\linewidth]{P1ReportDate.jpg}
    \caption[Date Command]{Date Command}
    \label{fig:datecommand}
  \end{figure}

  As expected, the xv6 date command prints the same information in the same format as the Linux date command and the seconds field is a few seconds later for the Linux command than the xv6 command.

  This test \textbf{PASSES}.


  \clearpage
  \subsection*{Control-P Format}
  In this test, I verified that Control-P displays processes with the correct
  header, that process information is aligned with the appropriate header, and
  that the correct data is displayed.

  It is expected that I will observe a well-formatted and correct output from
  the Control-P command.

  \begin{figure}[h!]
    \centering
    \includegraphics[width=0.8\linewidth]{P1ReportCtlP.jpg}
    \caption[Control-P output]{Control-p output}
    \label{fig:cpout}
  \end{figure}

  From the above figure, we can see that the header contains the appropriate
  fields, that the appropriate process information is aligned with each header
  item, and that the process information displayed is correct (it is assumed
  that the program counters are correct).

  The elapsed time for the {\tt init} process is slightly higher than for the
  {\tt sh} process. This is expected since the {\tt init} process is the first
  process to start and {\tt init} is responsible for starting {\tt sh}.

  In addition, the elapsed time increases with ever press of Control-P. This is
  the expected and desired behavior.

  This test \textbf{PASSES}.

\ifdefined \LF
} % large print end
\fi

\end{document}