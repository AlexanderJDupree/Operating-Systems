\documentclass[11pt,letterpaper]{report}
\usepackage[latin1]{inputenc}
\usepackage{amsmath}
\usepackage{amsfonts}
\usepackage{amssymb}
\usepackage{graphicx}
\usepackage{color}
\usepackage{enumitem}
\usepackage[dvipsnames]{xcolor}
\definecolor{codegray}{gray}{0.9}
\newcommand{\code}[1]{\colorbox{codegray}{\texttt{#1}}}
\graphicspath{{./images/}{IR}}
%\newcommand{\LF}{}  % turn on to display large format
\ifdefined \LF
\usepackage[left=2.0cm, top=2.0cm, landscape]{geometry}  % for large format landscape
\else
\usepackage[left=2.0cm, top=2.0cm]{geometry}
\fi
\usepackage{fancyhdr}
\pagestyle{fancy}
\fancyhead{}
\lhead{CS333}
\chead{Project 1 Test Report}
\rhead{Alexander DuPree}
\begin{document}
\title{Project 1 Test Report}
\author{Alexander DuPree}

\ifdefined \LF
{\Large     % large print start
\fi

  \maketitle
  \section*{Introduction}
  \noindent
  The following test report documents the tests performed for project one. The test cases and strategies closely follow the project one rubric. 

  Each section contains test cases related to the sections topic. Each test case will describe the name of the test, 
  the expected result, actual result, as well as a discussion and indication of the Pass/Fail status. 
  The actual result will be provided in the form of a screen shot of the console. 

  \section*{System Call Tracing}
  This section presents all tests related to tracing system calls within the xv6 system. 
  Test cases follow closely those outlined in the rubric. \hfill \break
  
  \noindent\textbf{Test Case:} \emph{With PRINT\_SYSCALLS set to 0 in the Makefile}
  
  \noindent\textbf{Assertions:}
  \begin{enumerate}[]
  \item Code correctly compiles
  \item No trace information is displayed
  \end{enumerate}  
  
  \noindent\textbf{Status:} \textcolor{ForestGreen}{\textbf{PASS}}
  
  \begin{figure}[h!]
	\centering
	\includegraphics[width=1\linewidth]{test1.png}
	\caption[PRINT\_SYSCALLS=0]{Compilation and execution with PRINT\_SYSCALLS set to 0.}
	\label{fig:P1compileP0-1}
  \end{figure}

  The command \code{grep "PRINT\_SYSCALLS ?=" Makefile} shows that the PRINT\_SYSCALLS macro is indeed turned off.
  The following command \code{make clean run} demonstrates that the code correctly compiles and the output does not display any trace information.
  Furthermore, it should be noted that the timestamps for each command was performed within thirty seconds of each other.
  
\pagebreak

  \noindent\textbf{Test Case:} \emph{With PRINT\_SYSCALLS set to 1 in the Makefile}
  
  \noindent\textbf{Assertions:}
  \begin{enumerate}[]
  \item Code correctly compiles
  \item Trace information is displayed
  \item Correct call trace on boot
  \end{enumerate}  
  
  \noindent\textbf{Status:} \textcolor{ForestGreen}{\textbf{PASS}} \hfill \break
  
  \begin{figure}[h!]
	\centering
	\includegraphics[height=\linewidth]{test2.png}
	\caption[PRINT\_SYSCALLS=0]{Compilation and Boot call trace with PRINT\_SYSCALLS set to 1.}
	\label{fig:P1compileP0-1}
  \end{figure}

  \pagebreak

  The command \code{grep "PRINT\_SYSCALLS ?=" Makefile} shows that the PRINT\_SYSCALLS macro is indeed turned on.
  The following command \code{make clean run} demonstrates that the code correctly compiles and the output displays
  trace information for the system calls. The system call trace in Figure 2 is the boot sequence for the xv6 system.
  Furthermore, it should be noted that the timestamps for each command was performed within 10 seconds of each other.

  \pagebreak

  \section*{Conditional Compilation and Date System Call}
  This section presents tests related to the conditional compilation of the CS333\_P1 project flag and the Date system call.
  Test cases follow closely those outlined in the rubric. \hfill \break
  
  \noindent\textbf{Test Case:} \emph{With CS333\_P1 turned off, when CS333\_PROJECT is set to 0}
  
  \noindent\textbf{Assertions:}
  \begin{enumerate}[]
  \item Code correctly compiles
  \item Kernel boots
  \item forktest runs to completion correctly
  \item usertests runs to completion correctly
  \end{enumerate}  
  
  \noindent\textbf{Status:} \textcolor{ForestGreen}{\textbf{PASS}}
  
  \begin{figure}[h!]
	\centering
	\includegraphics[width=1\linewidth]{test3.png}
	\caption[PRINT\_SYSCALLS=0]{Compilation and Kernel Boot with CS333\_P1 turned off.}
	\label{fig:P1compileP0-1}
  \end{figure}

  \begin{figure}[h!]
	\centering
	\includegraphics[width=1\linewidth]{test3-cont2.png}
	\caption[PRINT\_SYSCALLS=0]{Execution of usertests from the same session.}
	\label{fig:P1compileP0-1}
  \end{figure}

  \begin{figure}[h!]
	\centering
	\includegraphics[width=1\linewidth]{test3-cont.png}
	\caption[PRINT\_SYSCALLS=0]{Results of usertests with output elided.}
	\label{fig:P1compileP0-1}
  \end{figure}
  
  Figure 3, 4, and 5 present the results of this test case. The command \code{grep "CS333\_PROJECT ?=" Makefile}
  shows that the CS333\_PROJECT is defined with 0. The following command \code{make clean run} demonstrates that 
  the code correctly compiles the kernel successfully boots. Because \code{forktest} is included with the \code{usertests}
  only the user tests were ran. The results of which are shown in figure 4 and 5. Due to the size of the test, much of the 
  output was elided. 
  
\pagebreak
  
\ifdefined \LF
} % large print end
\fi

\end{document}